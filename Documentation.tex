\documentclass{report}
\usepackage{color}
\begin{document}
\tableofcontents

\chapter{Language Definition}
	\section{Syntax}
		\subsection{Identifiers}
			Identifiers follow the syntax defined for C. That is, each identifier is case sensitive. Each identifier must start with a letter( '\_' is also allowed). After this letters, underscores and/or integers may follow in any combination.
		\subsection{Keywords}
			\begin{itemize}
				\item "if"
				\item "return"
				\item "true"
				\item "false"
				%floating point types
				\item "float32"
				\item "float64"
				%boolean type
				\item "bool"
				\item "char"
				%integer types
				\item "int8"
				\item "uint8"
				\item "int16"
				\item "uint16"
				\item "int32"
				\item "uint32"
				\item "int"
				\item "uint"
				\item "int64"
				\item "uint64"
				
				\item "void"
			\end{itemize}

	\subsection{Integers}
		\paragraph{}
			Integers are represented by a sequence of numbers(obviously).

\chapter{Arithmetics / Data Operations}
	\section{Data Types}
		\paragraph{Integers}
		\paragraph{Floats}
		\paragraph{Other}
		\paragraph{\color{red}TODO}
			\color{red}
			\begin{itemize}
				\item Need to define standard integer types
				\item Need to define standard float types
				\item Need to define boolean type
				\item Need to define pointer type
				\item Need to define character type
				\item Need to define byte type
				\item Need to define type casting
			\end{itemize}
			\color{black}
	\section{Operations}
		


	\section{Order of Operations}

		\paragraph{\color{red}TODO}
			\color{red}
			\begin{itemize}
				\item Need to define order of operations
				\item Need to promotion and demotion of data types
			\end{itemize}
			\color{black}

\end{document}