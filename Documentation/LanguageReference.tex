\documentclass{standalone}
\usepackage{amsmath}
\usepackage{amsfonts}
\usepackage{amssymb}


 \newcommand{\tab}{\hspace*{2em}}
\begin{document}

\chapter{Language Reference}

	\section{Reserved Keywords}
		The following are keywords that are reserved.
		\begin{itemize}
			\item if
			\item else
			\item while
			\item print
			\item Integer
			\item Float
		\end{itemize}

	\section{Comments}
		\subsection{Line Comment}
			Line comments begin with a "//" and can appear any place on a line. Everything following  until a newline character is considered a comment and is ignored
		\subsection{Block Comment}
			Line comments begin with a "/*"  and end with a "*/". Everything in between is considered a comment and is ignored.

	\section{Constants}
		\subsection{Integer Constants}
			Integer constants are recognized as a contiguous sequence of digits.
		\subsection{Floating point Constants}
			Floating point constants are recognized as one of the following formats:
			\begin{enumerate}
				\item \{Sequence of digits\}.
				\item .\{Sequence of digits\}
				\item \{Sequence of digits\}.\{Sequence of digits\}
			\end{enumerate}
		\subsection{String Constants}
			Note strings are currently unsupported.
			Strings are recognized as any sequence of characters enclosed with double quotes") It should also be noted that strings can span multiple lines.

	\section{Types}
		Currently only Integers and Floats are supported. Integers and Floats have the same size as the compiler/interpreter's C long and double types, respectively. That is integers are implemented as C longs and floats are implemented as C doubles.

		\subsection{Booleans}
			Booleans are implicitly supported. That is any data item \footnote{ variable, constant or computational result} can be considered a boolean. A data item is considered to have a false value if all bits of the item are set to 0, otherwise it has a value of true.

		\subsection{Type Casting}
			Explicit type casting is not currently.
			Implicit type casting occurs in the following situations:\\
			\begin{tabular}{| l | p{5cm} | p{5cm}|}
				\hline
				Operation Type & Condition & Action Taken \\ 
				\hline
				Assignment	& type on both sides of the assignment don't match up & The data item is cast to the type of the variable being assigned to. \\ 
				\hline
				Mathematical Operation & types differ on both sides of the operator & Both sides of the operation are upcast to floats\\
				\hline
			\end{tabular}
	\section{Operators}
		The following table contains all available operators, their purpose and their implementation status:\\
		\begin{tabular}{| c | l | c |}
			\hline
			Operation & Symbol & Status\\
			\hline
			$+$ & Addition & \textcolor{green}{Available} \\
			\hline
			$-$ & Subtraction & \textcolor{green}{Available} \\
			\hline
			$*$ & Multiplication & \textcolor{green}{Available} \\
			\hline
			$/$ & Division & \textcolor{green}{Available} \\
			\hline
			$\%$ & Modulus & \textcolor{green}{Available} \\
			\hline
			\^{} & Exponentiation & \textcolor{green}{Available} \\
			\hline
			$=$ & Assignment & \textcolor{green}{Available} \\
			\hline
			$==$ & Equality Test & \textcolor{green}{Available} \\
			\hline
			$!=$ & non-Equality Test & \textcolor{green}{Available} \\
			\hline
			$>$ & Greater than Test & \textcolor{green}{Available} \\
			\hline
			$<$ & Less than Test & \textcolor{green}{Available} \\
			\hline
			$>=$ & Greater than or equal Test & \textcolor{green}{Available} \\
			\hline
			$<=$ & Greater than or equal Test & \textcolor{green}{Available} \\
			\hline
			$!$ & Logical Not & \textcolor{green}{Available} \\
			\hline
			$\&\&$ & Logical And & \textcolor{green}{Available} \\
			\hline
			$||$ & Logical Or & \textcolor{green}{Available} \\
			\hline
			$@@$ & Logical Xor & \textcolor{green}{Available} \\
			\hline
			\~{} & Bitwise Not & \textcolor{red}{Unavailable} \\
			\hline
			$\&$ & Bitwise And & \textcolor{red}{Unavailable} \\
			\hline
			$|$ & Bitwise Or & \textcolor{red}{Unavailable} \\
			\hline
			$@$ & Bitwise Xor & \textcolor{red}{Unavailable} \\
			\hline
			$<<$ & Left Bitwise Shift & \textcolor{red}{Unavailable} \\
			\hline
			$>>$ & Right Bitwise Shift & \textcolor{red}{Unavailable} \\
			\hline
		\end{tabular}

\end{document}
