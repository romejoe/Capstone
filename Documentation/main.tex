\documentclass{report}
\usepackage{amsmath}
\usepackage{amsfonts}
\usepackage{amssymb}
\usepackage{comment}
\usepackage{standalone}
\usepackage{hyperref}
\usepackage{color}

\newcommand{\tab}{\hspace*{2em}}

\hypersetup{linktocpage}


\begin{document}
\tableofcontents

\documentclass{standalone}
\usepackage{amsmath}
\usepackage{amsfonts}
\usepackage{amssymb}


 \newcommand{\tab}{\hspace*{2em}}
\begin{document}

\setlength{\grammarparsep}{20pt plus 1pt minus 1pt} % increase separation between rules
\setlength{\grammarindent}{12em} % increase separation between LHS/RHS 

\chapter{Language Reference}

	\section{Reserved Keywords}
		The following are keywords that are reserved.
		\begin{itemize}
			\item if
			\item else
			\item while
			\item print
			\item Integer
			\item Float
		\end{itemize}

	\section{Identifier Naming Conventions}
		Identifiers must start with a letter or `_' and can be followed by any combination of letters numbers or `_', including nothing. Identifiers are case-sensitive. Identifiers are currently only used for variables, but will be used for functions when the time comes. Below are some example identifiers:
		\begin{enumerate}
			\item `i'
			\item `I'
			\item `i1'
			\item `__i__'
			\item `_'
		\end{enumerate}
		All identifiers in the above list are distinguishable to the language.

	\section{Comments}
		\subsection{Line Comment}
			Line comments begin with a "//" and can appear any place on a line. Everything following  until a newline character is considered a comment and is ignored
		\subsection{Block Comment}
			Line comments begin with a "/*"  and end with a "*/". Everything in between is considered a comment and is ignored.

	\section{Constants}
		\subsection{Integer Constants}
			Integer constants are recognized as a contiguous sequence of digits.
		\subsection{Floating Point Constants}
			Floating point constants are recognized as one of the following formats:
			\begin{enumerate}
				\item \{Sequence of digits\}.
				\item .\{Sequence of digits\}
				\item \{Sequence of digits\}.\{Sequence of digits\}
			\end{enumerate}
		\subsection{String Constants}
			Note strings are currently unsupported.
			Strings are recognized as any sequence of characters enclosed with double quotes") It should also be noted that strings can span multiple lines.

	\section{Types}
		Currently only Integers and Floats are supported. Integers and Floats have the same size as the compiler/interpreter's C long and double types, respectively. That is integers are implemented as C longs and floats are implemented as C doubles.

		\subsection{Booleans}
			Booleans are implicitly supported. That is any data item \footnote{ variable, constant or computational result} can be considered a boolean. A data item is considered to have a false value if all bits of the item are set to 0, otherwise it has a value of true.

		\subsection{Type Casting}
			Explicit type casting is not currently.
			Implicit type casting occurs in the following situations:\\
			\begin{tabular}{| l | p{5cm} | p{5cm}|}
				\hline
				Operation Type & Condition & Action Taken \\ 
				\hline
				Assignment	& type on both sides of the assignment don't match up & The data item is cast to the type of the variable being assigned to. \\ 
				\hline
				Mathematical Operation & types differ on both sides of the operator & Both sides of the operation are up-cast to floats\\
				\hline
			\end{tabular}
	\section{Operators}
		The following table contains all available operators, their purpose and their implementation status:\\
		\begin{tabular}{| c | l | c |}
			\hline
			Symbol & Operation & Status\\
			\hline
			$+$ & Addition & \textcolor{green}{Available} \\
			\hline
			$-$ & Subtraction & \textcolor{green}{Available} \\
			\hline
			$*$ & Multiplication & \textcolor{green}{Available} \\
			\hline
			$/$ & Division & \textcolor{green}{Available} \\
			\hline
			$\%$ & Modulus & \textcolor{green}{Available} \\
			\hline
			\^{} & Exponentiation & \textcolor{green}{Available} \\
			\hline
			$=$ & Assignment & \textcolor{green}{Available} \\
			\hline
			$==$ & Equality Test & \textcolor{green}{Available} \\
			\hline
			$!=$ & non-Equality Test & \textcolor{green}{Available} \\
			\hline
			$>$ & Greater than Test & \textcolor{green}{Available} \\
			\hline
			$<$ & Less than Test & \textcolor{green}{Available} \\
			\hline
			$>=$ & Greater than or equal Test & \textcolor{green}{Available} \\
			\hline
			$<=$ & Greater than or equal Test & \textcolor{green}{Available} \\
			\hline
			$!$ & Logical Not & \textcolor{green}{Available} \\
			\hline
			$\&\&$ & Logical And & \textcolor{green}{Available} \\
			\hline
			$||$ & Logical Or & \textcolor{green}{Available} \\
			\hline
			$@@$ & Logical Xor & \textcolor{green}{Available} \\
			\hline
			\~{} & Bitwise Not & \textcolor{red}{Unavailable} \\
			\hline
			$\&$ & Bitwise And & \textcolor{red}{Unavailable} \\
			\hline
			$|$ & Bitwise Or & \textcolor{red}{Unavailable} \\
			\hline
			$@$ & Bitwise Xor & \textcolor{red}{Unavailable} \\
			\hline
			$<<$ & Left Bitwise Shift & \textcolor{red}{Unavailable} \\
			\hline
			$>>$ & Right Bitwise Shift & \textcolor{red}{Unavailable} \\
			\hline
		\end{tabular}

	\section{Variables}
		\subsection{Variable Naming Conventions}
			Variable names
		\subsection{Definition}
			Variable definitions follow the following grammar definition:
			\begin{grammar}
				<variable-def> ::= <type> <varname>
					\alt <type> <varname> `=' <expression>'

				<type> ::= `Integer'
					\alt `Float'
			\end{grammar}
			`varname' must follow the identifier naming conventions.
		\subsection{Usage}
			Variables are used by simply using their name.
	
	\section{Statements}
		Statements are broken down into 2 categories: general and control. General statements must always end with a `;'. Control statements in general will not end in a `;'.
		Below is a list of available general statements:
		\subsection{General statements}
		\begin{grammar}
			<general-statement> ::= <variable-def> `;'
				\alt <variable> `=' <expression> `;'
				\alt `print' <expression> `;'
				\alt <expression> `;'
		\end{grammar}
		\subsubsection{Print statement}
			Printing variables is supported via the print keyword. The type and value of the expression following `print' is output.

		\subsection{Control statements}
		Below is a list of available control statements:
		\begin{grammar}
			<control-statement> ::= 
				`if' `(' <expression> `)' `{' <statement-group> `}'
				\alt `if' `(' <expression> `)' `{' <statement-group> `}' `else' `{' <statement-group> `}'
				\alt `while' `(' <expression> `)' `{' <statement-group> `}'
		
			<statement-group> ::= <statement-group> <statement>
				\alt <statement>

			<statement> ::= <general-statement>
				\alt <control-statement>
	
		\end{grammar}

		

	\section{Scope}
		Scope level of code increases with every pair of `\{' `\}' enclosing it. Any code not enclosed is considered global code. It follows that variables defined in global code are global variables, with all other variable definitions defining local variables.

		Variables are available to all scopes that they are defined above. Should a variable defined in a local scope take the same name as a variable in a higher scope, the local variable is still used for that scope and all lower scopes.

\end{document}


\chapter{Compiler}
	\subsection{Lexer}
		The
	\subsection{Lemon}


\documentclass{standalone}
\usepackage{amsmath}
\usepackage{amsfonts}
\usepackage{amssymb}

 \newcommand{\tab}{\hspace*{2em}}
\begin{document}

\chapter{Interpreter Reference}
	\section{Executable Format}
		Currently executables consist of only code. The interpreter simply starts at the beginning of the file and starts executing.

		\subsection{Proposed Executable Format}
			The current file format is reaching its limits. The following features are probably out of reach of the current format:
			\begin{enumerate}
				\item Different object types: Shared Code and Executable Code
				\item 32/64 bit executables
				\item String data
			\end{enumerate}
			I am therefor proposing the Executable and Linkable Format(ELF) as a good starting point, if not complete solution.

	\section{Execution PseudoCode}
		The interpreter follows the following process for execution.
		\begin{enumerate}
			\item Initialize computational stack and variable list 
			\item While instruction pointer doesn't equal the end of the instruction buffer
				\begin{enumerate}
					\item Retrieve instruction
					\item Execute instruction (see the ByteCode Reference for the function of each instruction)
				\end{enumerate}
		\end{enumerate}

\end{document}

\documentclass{standalone}
\usepackage{amsmath}
\usepackage{amsfonts}
\usepackage{amssymb}

 \newcommand{\tab}{\hspace*{2em}}
\begin{document}

\chapter{Assembly/Byte Code Language Reference}
	All instructions in the defined in an enum, defined in "Common/Instruction.h".
	All here are all of the assemble language instructions that are currently supported (Note, the parameters/output for any given instruction represent how the top of the computational stack should look like before/after the instruction executes) :

\subsection{iHello}
\begin{tabular}{l l p{10cm}}
Description: & ~    & This is a debug instruction. Causes interpreter to print out "Hello First Instruct"\\
Parameters:  & None & ~ \\
Output:      & None & ~ \\
\end{tabular}


\subsection{iJMP}
\begin{tabular}{l l p{10cm}}
Description: & ~    & Does a relative jump. Offset is added to the instruction pointer.\\
Parameters:  & ~ & ~ \\
~			 & Offset: & Offset to jump by.\\
Output:      & None & ~ \\
\end{tabular}

\subsection{iJMPT}
\begin{tabular}{l l p{10cm}}
Description: & ~    & Does a relative jump if condition is true. Offset is added to the instruction pointer.\\
Parameters:  & ~ & ~ \\
~			 & Offset: & Offset to jump by.\\
~			 & Condition: & Result of Conditional Statement.\\
Output:      & None & ~ \\
\end{tabular}

\subsection{iJMPF}
\begin{tabular}{l l p{10cm}}
Description: & ~    & Does a relative jump if condition is false. Offset is added to the instruction pointer.\\
Parameters:  & ~ & ~ \\
~			 & Offset: & Offset to jump by.\\
~			 & Condition: & Result of Conditional Statement.\\
Output:      & None & ~ \\
\end{tabular}

\subsection{iEQ}
\begin{tabular}{l l p{10cm}}
Description: & ~    & Compares the bits of LHS and RHS\\
Parameters:  & ~ & ~ \\
~			 & RHS: & Result of prior computation\\
~			 & LHS: & Result of prior computation\\
Output:      & Result & true if all bits of LHS are the same as RHS, false otherwise                        \\
\end{tabular}

\subsection{iNEQ}
\begin{tabular}{l l p{10cm}}
Description: & ~    & Compares the bits of LHS and RHS\\
Parameters:  & ~ & ~\\
~			 & RHS: & Result of prior computation\\
~			 & LHS: & Result of prior computation\\
Output:      & Result & false if all bits of LHS are the same as RHS, true otherwise\\
\end{tabular}

\subsection{iLT}
\begin{tabular}{l l p{10cm}}
Description: & ~    & Casts RHS and LHS to appropriate types (See typecast guide) and returns true if LHS $<$ RHS\\
Parameters:  & ~ & ~ \\
~			 & RHS: & Result of prior computation\\
~			 & LHS: & Result of prior computation\\
Output:      & Result & true if LHS $<$ RHS, false otherwise\\
\end{tabular}

\subsection{iLTE}
\begin{tabular}{l l p{10cm}}
Description: & ~    & Casts RHS and LHS to appropriate types (See typecast guide) and returns true if LHS $<=$ RHS\\
Parameters:  & ~ & ~ \\
~			 & RHS: & Result of prior computation\\
~			 & LHS: & Result of prior computation\\
Output:      & Result & true if LHS $<=$ RHS, false otherwise\\
\end{tabular}

\subsection{iGT}
\begin{tabular}{l l p{10cm}}
Description: & ~    & Casts RHS and LHS to appropriate types (See typecast guide) and returns true if LHS $>$ RHS\\
Parameters:  & ~ & ~ \\
~			 & RHS: & Result of prior computation\\
~			 & LHS: & Result of prior computation\\
Output:      & Result & true if LHS $>$ RHS, false otherwise\\
\end{tabular}

\subsection{iGTE}
\begin{tabular}{l l p{10cm}}
Description: & ~    & Casts RHS and LHS to appropriate types (See typecast guide) and returns true if LHS $>=$ RHS\\
Parameters:  & ~ & ~ \\
~			 & RHS: & Result of prior computation\\
~			 & LHS: & Result of prior computation\\
Output:      & Result & true if LHS $>=$ RHS, false otherwise\\
\end{tabular}

\subsection{iADD}
\begin{tabular}{l l p{10cm}}
Description: & ~    & Casts RHS and LHS to appropriate types (See typecast guide) and returns LHS + RHS\\
Parameters:  & ~ & ~ \\
~			 & RHS: & Result of prior computation\\
~			 & LHS: & Result of prior computation\\
Output:      & Result & LHS + RHS\\
\end{tabular}

\subsection{iSUB}
\begin{tabular}{l l p{10cm}}
Description: & ~    & Casts RHS and LHS to appropriate types (See typecast guide) and returns LHS - RHS\\
Parameters:  & ~ & ~ \\
~			 & RHS: & Result of prior computation\\
~			 & LHS: & Result of prior computation\\
Output:      & Result & LHS - RHS\\
\end{tabular}

\subsection{iMUL}
\begin{tabular}{l l p{10cm}}
Description: & ~    & Casts RHS and LHS to appropriate types (See typecast guide) and returns LHS * RHS\\
Parameters:  & ~ & ~ \\
~			 & RHS: & Result of prior computation\\
~			 & LHS: & Result of prior computation\\
Output:      & Result & LHS * RHS\\
\end{tabular}

\subsection{iDIV}
\begin{tabular}{l l p{10cm}}
Description: & ~    & Casts RHS and LHS to appropriate types (See typecast guide) and returns LHS / RHS\\
Parameters:  & ~ & ~ \\
~			 & RHS: & Result of prior computation\\
~			 & LHS: & Result of prior computation\\
Output:      & Result & LHS / RHS\\
\end{tabular}

\subsection{iMOD}
\begin{tabular}{l l p{10cm}}
Description: & ~    & Casts RHS and LHS to integers and returns LHS (mod RHS)\\
Parameters:  & ~ & ~ \\
~			 & RHS: & Result of prior computation\\
~			 & LHS: & Result of prior computation\\
Output:      & Result & LHS (mod RHS)\\
\end{tabular}

\subsection{iPOW}
\begin{tabular}{l l p{10cm}}
Description: & ~    & Casts BASE and EXP to floats and returns $\text{BASE}^{\text{EXP}}$\\
Parameters:  & ~ & ~ \\
~			 & EXP: & Result of prior computation\\
~			 & BASE: & Result of prior computation\\
Output:      & Result & $\text{BASE}^{\text{EXP}}$\\
\end{tabular}

\subsection{iIPUSH}
\begin{tabular}{l l p{10cm}}
Description: & ~    & Interprets the next N bytes of the program buffer as an integer and pushes the value onto the computational stack, where N is the interpreter's defined size of an integer\\
Parameters:  & None & ~ \\
Output:      & Result & Integer following the instruction\\
\end{tabular}

\subsection{iFPUSH}
\begin{tabular}{l l p{10cm}}
Description: & ~    & Interprets the next N bytes of the program buffer as a float and pushes the value onto the computational stack, where N is the interpreter's defined size of a float\\
Parameters:  & None & ~ \\
Output:      & Result & Float following the instruction\\
\end{tabular}

\subsection{iLVPUSH}
\begin{tabular}{l l p{10cm}}
Description: & ~    & Interprets the next N bytes of the program buffer as an integer offset and pushes the variable with the offset relative to the local variable pointer, where N is the interpreter's defined size of a integer \\
Parameters:  & None & ~ \\
Output:      & Result & Variable at index\\
\end{tabular}

\subsection{iGVPUSH}
\begin{tabular}{l l p{10cm}}
Description: & ~    & Interprets the next N bytes of the program buffer as an integer offset and pushes the variable with the offset relative to the global variable pointer, where N is the interpreter's defined size of a integer \\
Parameters:  & None & ~ \\
Output:      & Result & Variable at index\\
\end{tabular}

\subsection{iASSIGN}
\begin{tabular}{l l p{10cm}}
Description: & ~    & Assigns Result to Target\\
Parameters:  & ~ & ~ \\
~			 & Result: & Result of prior computation\\
~			 & Target: & Variable Item\\
Output:      & ~ & ~\\
\end{tabular}

\subsection{iVALLOC}
\begin{tabular}{l l p{10cm}}
Description: & ~    & Interprets the next N bytes of the program buffer as an integer amount and allocates that many raw variables onto the variable stack, where N is the interpreter's defined size of a integer \\
Parameters:  & None & ~ \\
Output:      & ~ & ~\\
\end{tabular}

\subsection{iVDALLOC}
\begin{tabular}{l l p{10cm}}
Description: & ~    & Interprets the next N bytes of the program buffer as an integer amount and deallocates that many raw variables from the variable stack, where N is the interpreter's defined size of a integer \\
Parameters:  & None & ~ \\
Output:      & ~ & ~\\
\end{tabular}

\subsection{iVSETTYPE}
\begin{tabular}{l l p{10cm}}
Description: & ~    & Interprets the next N bytes of the program buffer as an integer offset, Interprets the next M bytes of the program buffer as a datasource and sets the variable with the offset relative to the local variable pointer to that type, where N is the interpreter's defined size of a integer and M is the interpreter's defined size of a datasource enum\\
Parameters:  & None & ~ \\
Output:      & ~ & ~\\
\end{tabular}

\subsection{iDUMPVARS}
\begin{tabular}{l l p{10cm}}
Description: & ~    & This is a debug instruction. Prints all variables in the variable list\\
Parameters:  & None & ~ \\
Output:      & ~ & ~\\
\end{tabular}

\subsection{iLOGNOT}
\begin{tabular}{l l p{10cm}}
Description: & ~    & Casts RHS to boolean (See typecast guide) and returns the inverse of RHS\\
Parameters:  & ~ & ~ \\
~			 & RHS: & Result of prior computation\\
Output:      & Result & inverse of RHS\\
\end{tabular}

\subsection{iLOGAND}
\begin{tabular}{l l p{10cm}}
Description: & ~    & Casts RHS and LHS to booleans (See typecast guide) and returns true if LHS $=$ RHS\\
Parameters:  & ~ & ~ \\
~			 & RHS: & Result of prior computation\\
~			 & LHS: & Result of prior computation\\
Output:      & Result & true if LHS $=$ RHS, false otherwise\\
\end{tabular}

\subsection{iLOGOR}
\begin{tabular}{l l p{10cm}}
Description: & ~    & Casts RHS and LHS to booleans (See typecast guide) and returns true if LHS or RHS or (LHS and RHS) are true (i.e. standard logical or)\\
Parameters:  & ~ & ~ \\
~			 & RHS: & Result of prior computation\\
~			 & LHS: & Result of prior computation\\
Output:      & Result & standard logical or\\
\end{tabular}

\subsection{iLOGXOR}
\begin{tabular}{l l p{10cm}}
Description: & ~    & Casts RHS and LHS to booleans (See typecast guide) and returns true if LHS or RHS are true (i.e. standard logical xor)\\
Parameters:  & ~ & ~ \\
~			 & RHS: & Result of prior computation\\
~			 & LHS: & Result of prior computation\\
Output:      & Result & standard logical xor\\
\end{tabular}

\subsection{iPRINT}
\begin{tabular}{l l p{10cm}}
Description: & ~    & Prints Result\\
Parameters:  & ~ & ~ \\
~			 & Result & Result of prior computation\\
Output:      & ~ & ~\\
\end{tabular}

\end{document}



\begin{comment}
\chapter{Interpreter}
	\subsection{Executable Format}
	Currently the executable is just code. The interpreter simply starts at the beginning of the file and starts executing.
	This file format is reaching its limits. To avoid a huge mess, an alternate file format is going to be needed. 
	\subsection{Model}
	The interpreter, rather than being register based, is stack based. I went with a stack-based mode primarily for its ease of use and ease of development. 
\end{comment}


\begin{comment}
\chapter{Model of Computation}
	The interpreter is currently equivalent to a push down automata (it may be more powerful, but I'm not sure how to prove it). However, the compiler and language are currently equivalent to a finite state machine. This is do to the grammar not providing a mechanism for modifying the computational stack.
\end{comment}



\end{document}