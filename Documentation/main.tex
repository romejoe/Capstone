\documentclass{report}
 \newcommand{\tab}{\hspace*{2em}}

\begin{document}
\tableofcontents
\chapter{Language Spec}
	\subsection{Keywords}
		\begin{enumerate}
			\item if
			\item else
			\item while
			\item print
			\item Integer
			\item Float
		\end{enumerate}
	
\chapter{Compiler}
	\subsection{Lexer}
		The
	\subsection{Lemon}
\chapter{Interpreter}
	\subsection{Executable Format}
	Currently the executable is just code. The interpreter simply starts at the beginning of the file and starts executing.
	This file format is reaching its limits. To avoid a huge mess, an alternate file format is going to be needed. 
	\subsection{Model}
	The interpreter, rather than being register based, is stack based. I went with a stack-based mode primarily for its ease of use and ease of development. 


\chapter{Model of Computation}
	The interpreter is currently equivalent to a push down automata (it may be more powerful, but I'm not sure how to prove it). However, the compiler and language are currently equivalent to a finite state machine. This is do to the grammar not providing a mechanism for modifying the computational stack.
\end{document}