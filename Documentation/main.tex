\documentclass{report}
 \newcommand{\tab}{\hspace*{2em}}

\begin{document}
\tableofcontents
\chapter{Language Spec}
	\subsection{Keywords}
		\begin{enumerate}
			\item if
			\item else
			\item while
			\item print
			\item Integer
			\item Float
		\end{enumerate}
	
\chapter{Compiler}
	\subsection{Lexer}
		The
	\subsection{Lemon}
\chapter{Interpreter}
	\subsection{Executable Format}
	Currently the executable is just code. The interpreter simply starts at the beginning of the file and starts executing.
	This file format is reaching its limits. To avoid a huge mess, an alternate file format is going to be needed. 
	\subsection{Model}
	The interpreter, rather than being register based, is stack based. I went with a stack-based mode primarily for its ease of use and ease of development. 

\chapter{Assembly/Byte Code Language Reference}
	All instructions in the defined in an enum, defined in "Common/Instruction.h".
	All here are all of the assemble language instructions that are currently supported (Note, the parameters/output for any given instruction represent how the top of the computational stack should look like before/after the instruction executes) :

	\subsection{iHello}
\begin{tabular}{l l p{10cm}}
Description: & ~    & This is a debug instruction. Causes interpreter to print out "Hello First Instruct"\\
Parameters:  & None & ~                                                                                   \\
Output:      & None & ~                                                                                   \\
\end{tabular}


\subsection{iJMP}
\begin{tabular}{l l p{10cm}}
Description: & ~    & Unconditionally does a relative jump. Offset is added to the instruction pointer.\\
Parameters:  & ~ & ~                                                                                   \\
~			 & Offset: & Offset to jump by.\\
Output:      & None & ~                                                                                   \\
\end{tabular}

\subsection{iJMPT}
\begin{tabular}{l l p{10cm}}
Description: & ~    & Does a relative jump if condition is true. Offset is added to the instruction pointer.\\
Parameters:  & ~ & ~                                                                                   \\
~			 & Offset: & Offset to jump by.\\
~			 & Condition: & Result of Conditional Statement.\\
Output:      & None & ~                                                                                   \\
\end{tabular}

\subsection{iJMPF}
\begin{tabular}{l l p{10cm}}
Description: & ~    & Does a relative jump if condition is false. Offset is added to the instruction pointer.\\
Parameters:  & ~ & ~                                                                                   \\
~			 & Offset: & Offset to jump by.\\
~			 & Condition: & Result of Conditional Statement.\\
Output:      & None & ~                                                                                   \\
\end{tabular}

\subsection{iEQ}
\begin{tabular}{l l p{10cm}}
Description: & ~    & Compares the bits of LHS and RHS\\
Parameters:  & ~ & ~                                                                                   \\
~			 & RHS: & Result of computation\\
~			 & LHS: & Result of computation\\
Output:      & Result & true if all bits of LHS are the same as RHS, false otherwise                        \\
\end{tabular}


\chapter{Model of Computation}
	The interpreter is currently equivalent to a push down automata (it may be more powerful, but I'm not sure how to prove it). However, the compiler and language are currently equivalent to a finite state machine. This is do to the grammar not providing a mechanism for modifying the computational stack.
\end{document}